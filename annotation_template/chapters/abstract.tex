\begin{abstract}
% skip one line to make the abstract start with indent

Все более широкое использование гетерогенных мобильных платформ заставляет разрабатывать более эффективные планировщики. Такие платформы состоят из вычислительных устройств, различающихся по времени работы и энергопотреблению. Задача планировщика - найти подходящий компромисс между временем выполнения и энергопотреблением. Мы предлагаем новые оптимальные и субоптимальные методы составления расписания, основанные на понятии графов переходов состояний, которые моделируют возможные расписания выполнения программ на различных типах вычислительных устройств. В качестве цели планирования мы выбрали метрику Energy-Delay Product, которая объединяет время и энергопотребление. Наши планировщики реализованы с помощью инструмента C++ Scheduler, который доступен по запросу.
\end{abstract}