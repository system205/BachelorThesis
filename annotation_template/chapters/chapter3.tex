\chapter{Методология}
\label{chap:met}

Далее приведем игрушечный пример планирования трех программ на гетерогенной аппаратной платформе с одним ``медленным'' и одним ``быстрым'' ядром. Мы предполагаем, что программы состоят из отдельных логически разделенных блоков кода\footnote{Понятие блока еще не определено и будет уточнено позже. Пока мы предполагаем, что блоки программы выполняются последовательно}, а требования к их выполнению перечислены в таблице~\ref{tab:demandExample}. На рис.~\ref{fig:s1} и~\ref{fig:s2} мы изобразили загрузку ядер и выполнение программ для двух возможных расписаний S1 и S2, предполагая, что программы поступают асинхронно в моменты времени 0, 1 и 3 соответственно. Например, в S1 ``медленное'' ядро 1 выделено для выполнения программ 1 и 3, а в S2 это ядро выполняет только все блоки программы 2.
