\chapter{Выводы}
\label{chap:conclusion}

Мы решили проблему эффективного планирования работы гетерогенных систем с учетом времени отклика программ и энергопотребления. Мы предложили несколько планировщиков, среди которых есть оптимальный и эвристический: случайный и жадный. Наш ключевой вклад - учет энергопотребления, в отличие от других существующих планировщиков, таких как HEFT и HASS.


\section{Детальный обзор вклада}
В основе наших планировщиков лежит обход графа переходов состояний программ. Оптимальный планировщик должен обойти весь огромный по размеру граф, что требует значительного времени вычислений. Эвристики обходят только выбранные фрагменты графа, стремясь сократить время вычислений, но создают менее эффективные расписания. Чтобы проверить эффективность планировщиков, мы реализовали инструмент C++ Планировщик для моделирования параллельного выполнения входных программ. Также мы реализовали инструмент для генерации параметров входных программ для проведения экспериментов. Эффективность планировщиков оценивается в ходе серии экспериментов в терминах произведения энергии и задержки (EDP), которое учитывает как время отклика программ, так и потребляемую энергию. Мы показали, что эвристика 100-случайных прогулок близка к оптимальному расписанию, в то время как жадная эвристика показывает наихудшую эффективность. Что касается времени выполнения, то вычисление оптимального расписания занимает на порядки больше времени из-за экспоненциальной сложности по сравнению с эвристиками. Отклонение EDP для расписаний составляет 40-50\% для выбранных экспериментальных настроек. Мы пока не учитываем накладные расходы на миграцию и преемственность, которые должны быть включены.

Инструменты C++, используемые в данной работе доступно по требованию.