\documentclass[oneside,final,14pt,a4paper]{extreport}

\usepackage{tempora}   

\usepackage{vmargin}
\setpapersize{A4}
\setmarginsrb{2.5cm}{2cm}{2cm}{2cm}{0pt}{10mm}{0pt}{13mm}
\usepackage{setspace}
\sloppy
\setstretch{1.5}
\usepackage{indentfirst}
\parindent=1.25cm

%%%%% ADDED TO SUPPORT TT BOLD FACES %%%%
\DeclareFontShape{OT1}{cmtt}{bx}{n}{<5><6><7><8><9><10><10.95><12><14.4><17.28><20.74><24.88>cmttb10}{}
\renewcommand{\ttdefault}{pcr}
%%%%% END %%%%%%%%%%%%%%%%%%%%%%%%%%%%%%% 
\usepackage{atbegshi,picture}
\usepackage[T1,T2A]{fontenc} 
\usepackage[utf8]{inputenc}
\usepackage[main=russian,english]{babel}
\usepackage[backend=biber,style=ieee,autocite=inline]{biblatex}
\bibliography{ref.bib}
\usepackage{csquotes}
\usepackage{blindtext}


\usepackage{pdfpages}
\newenvironment{bottompar}{\par\vspace*{\fill}}{\clearpage}

% \usepackage{cite}
\usepackage{amsmath,amsfonts}
\usepackage{amsthm}
\newtheorem{theorem}{Theorem}
\newtheorem{corollary}{Corollary}
\newtheorem{lemma}{Lemma}
\newtheorem{proposition}{Proposition}
\theoremstyle{definition}
\newtheorem{definition}{Definition}
\theoremstyle{remark}
\newtheorem*{remark}{Remark}
\theoremstyle{remark}
\newtheorem*{example}{Example}



\usepackage{graphicx}
\graphicspath{{figs/}} %path to images
\usepackage{multirow,array}
\usepackage{caption}
\usepackage{subcaption}
\usepackage[unicode]{hyperref}
\hypersetup{colorlinks=true, linkcolor=black, citecolor=black}
\usepackage{paralist}
\usepackage{listings}
\usepackage{zed-csp}
\usepackage{fancyhdr}
\usepackage{color,colortbl}
\usepackage{booktabs}
\usepackage{epsfig} % for postscript graphics files

\usepackage{upgreek} 
\usepackage{bm}
\usepackage{hyperref}
\usepackage{longtable}
\usepackage{makecell}
\usepackage[font=singlespacing, labelfont=bf, justification=raggedright, singlelinecheck = false]{caption}
\usepackage{floatrow}

\pagestyle{fancyplain}

% remember section title
\renewcommand{\chaptermark}[1]%
	{\markboth{\chaptername~\thechapter~--~#1}{}}

% subsection number and title
\renewcommand{\sectionmark}[1]%
	{\markright{\thesection\ #1}}

\rhead[\fancyplain{}{\bf\leftmark}]%
      {\fancyplain{}{\bf\thepage}}
\lhead[\fancyplain{}{\bf\thepage}]%
      {\fancyplain{}{\bf\rightmark}}
\cfoot{} %bfseries


\newcommand{\dedication}[1]
   {\thispagestyle{empty}
     
   \begin{flushleft}\raggedleft #1\end{flushleft}
}

\begin{document}

\includepdf[offset=2.5cm -2cm]{title.pdf}

\newpage
\tableofcontents
\begin{abstract}
% skip one line to make the abstract start with indent

Моя аннотация начинается здесь.
\end{abstract}
\setcounter{page}{4}
% set manually the number, from which Глава 1 starts!
% Why do we put 4 in this case?
% Title page - page 1
% Оглавление - page 2
% Аннотация - page 3
% Глава 1 - page 4
% In your annotation the counter number can be different, please count carefully and insert the corresponding number.

\chapter{Введение}
\label{chap:intro}

В классическом однородном многоядерном планировании планировщик операционной системы распределяет доступные однородные ядра между динамически поступающими программами~\cite{Anderson2006}. Когда количество ожидающих выполнения программ превышает количество ядер, планировщик распределяет программы по приоритетам, либо по статическим приоритетам, установленным разработчиком системы, либо по конкретным динамически вычисляемым приоритетам. Целями планировщика являются минимизация среднего времени выполнения программ, задержки, числа миграций и вытеснений и другие. Мы подчеркиваем, что проблема оптимального планирования работы многоядерных систем является $\mathsf{NP}$-трудной, и поэтому эвристические подходы обычно используются для поиска субоптимального решения.

\section{Гетерогенное планирование}
Кроме того, в отличие от классического гомогенного случая, гетерогенное планирование значительно усложняется наличием нескольких типов вычислительных блоков для гетерогенной аппаратной платформы~\cite{Radulescu2000}. Некоторые из этих гетерогенных блоков медленные, но энергоэффективные, а другие быстрые, но слишком энергозатратные (см. рис.~\ref{fig:heteroExampleOverview}). Известным примером гетерогенной платформы является архитектура ARM big.LITTLE~\cite{Padoin2015}, концептуально изображенная на рис.~\ref{fig:bigLITTLEArchitecture}. Она состоит из:
\begin{itemize}
\item кластер быстрых (``больших''), но энергоемких ядер;
\item кластер медленных (``маленьких''), но энергоэффективных ядер;
\item GPU и
\item устройства NPU.
\end{itemize}
Посмотрите на сравнение производительности ``больших'' (Cortex-A15) и ``маленьких'' (Cortex-A7) ядер на рис.~\ref{fig:CortexA15A7Comparison}. Для этих типов ядер существует пересекающийся диапазон производительности, в котором использование ``маленьких'' ядер приводит к гораздо меньшему энергопотреблению, чем использование ``больших'' ядер.
Соответствующее распределение этих вычислительных блоков зависит от активности программ. Например, чтение 32 байт из 8-Кбайт SRAM потребляет в 125 раз меньше энергии, чем из DRAM~\cite{Hennessy2017}
%, в то время как чтение из SRAM потребляет в 50 раз больше энергии, чем 32-битное целочисленное сложение. 
Таким образом, гетерогенный планировщик направлен не только на оптимизацию временных аспектов, но и на оптимизацию общего энергопотребления. Эта проблема оптимизации гетерогенного планирования особенно актуальна для систем с автономными источниками питания.

\begin{figure}
\centering
\includegraphics[width=0.5\columnwidth]{figs/heteroExampleOverview.pdf}
\caption{Ключевые компоненты гетерогенного планирования}
\label{fig:heteroExampleOverview}
\end{figure}

\begin{table}
\centering
\captionsetup{justification=centering}
\caption{Требования исполнения программ по их блокам}
\label{tab:demandExample}
\begin{tabular}{|c|c|c|c|c|c|}
\hline
\multicolumn{2}{|c|}{\multirow{2}{*}[-1ex]{\makecell{Программы \\ Блоки}}} & \multicolumn{2}{c|}{Время, сек} & \multicolumn{2}{c|}{Энергия, Джоули} \\[2pt] \cline{3-6}
\multicolumn{2}{|c|}{} & \makecell{Ядро 1 \\ (медленное)} & \makecell{Ядро 2 \\ (быстрое)} & Ядро 1 & Ядро 2 \\ \hline
\multirow{2}{*}{P1} & B1 & 4 & 3 & 8 & 9 \\[2pt] %\cline{2-6}
 & B2 & 5 & 3 & 3 & 4 \\[2pt] \hline
\multirow{3}{*}{P2} & B1 & 5 & 4 & 6 & 10 \\[2pt] %\cline{2-6}
 & B2 & 3 & 2 & 3 & 7 \\[2pt] %\cline{2-6}
 & B3 & 3 & 2 & 3 & 6 \\[2pt] \hline
P3 & B1 & 3 & 2 & 6 & 7 \\[2pt] %\hline
 \hline
\end{tabular}
\end{table}

\begin{figure*}
\begin{subfigure}{.85\columnwidth}
\includegraphics[width=1\linewidth]{figs/bigLITTLEArchitecture.pdf}
\vspace{1mm}
\caption{ARM big.LITTLE архитектура: обзор}
\label{fig:bigLITTLEArchitecture}
\end{subfigure}

\begin{subfigure}{.95\columnwidth}
\includegraphics[width=.4\linewidth]{figs/CortexA15vsA7Comparison.pdf}
\vspace{1mm}
\caption{Тренды в производительности и энергопотреблении}
\label{fig:CortexA15A7Comparison}
\end{subfigure}
\caption{Гетерогенная архитектура: пример Cortex-A15 и Cortex-A7}
\end{figure*}

\begin{figure}
\centering
\includegraphics[width=.18\columnwidth]{figs/notation.pdf}
\caption{Графическая нотация}
\label{fig:notation}
\end{figure}


Рисунки, как и остальная часть работы, опираются на графические обозначения, приведенные на рис.~\ref{fig:notation}.

\begin{figure*}
\begin{minipage}{.7\columnwidth}%
\begin{subfigure}{\linewidth}
\includegraphics[width=\linewidth]{figs/s1.pdf}
\caption{Schedule S1: исполнение программ}
\vspace{3mm}
\label{fig:s1}
\end{subfigure}

\begin{subfigure}{\linewidth}
\centering
\includegraphics[width=\linewidth]{figs/s1Metrics.pdf}
\caption{S1: временные метрики}
\vspace{1mm}
\label{fig:s1Metrics}
\end{subfigure}
\end{minipage}

\begin{minipage}{.7\columnwidth}%
\begin{subfigure}{\linewidth}
\includegraphics[width=\linewidth]{figs/s2.pdf}
\caption{Schedule S2: исполнение программ}
\vspace{3mm}
\label{fig:s2}
\end{subfigure}

\begin{subfigure}{\linewidth}
\includegraphics[width=\linewidth]{figs/s2Metrics.pdf}
\caption{S2: временные метрики}
\vspace{1mm}
\label{fig:s2Metrics}
\end{subfigure}
\end{minipage}%

\caption{Примеры планирования программ для требований из таблицы~\ref{tab:demandExample} (миграции и преемственность не учтены)}
\label{fig:twoSchedulesExample}
\vspace{-3mm}
\end{figure*}





















%\chapter{System Model}
\label{chap:systemModel}
\chaptermark{Second Chapter Heading}

Let us formally describe the necessary definitions and assumptions for an addressed heterogeneous scheduling problem. We consider scheduling over ARM \textsf{\textsc{big}}.\textsf{LITTLE} platform, which is introduced in Chapter~\ref{chap:intro}. However, to keep our analysis simpler, for now we restrict the platform to the case of two cores only:
%
\begin{itemize}
\item The \textsf{\textsc{big}} core \-- fast but energy-hungry;
\item The \textsf{LITTLE} core \-- slow but energy-efficient;
\end{itemize}
%
We assume every core to execute only one program at a time. We also consider no hardware-level parallelism and other optimisations at the moment.

\section{Mathematical notations}

We have $N$ programs, which are denoted by $P_1,\ldots,P_N$, to be executed concurrently over a shared \textsf{\textsc{big}}.\textsf{LITTLE} platform of $\mathcal{M}$ cores\footnote{In our analysis, $\mathcal{M}=2$: one \textsf{\textsc{big}} and one \textsf{LITTLE} cores}. These programs arrive asynchronously into a pending queue as depicted in Fig.~\ref{fig:heteroExampleOverview}. Then, at scheduling time instants a heterogeneous scheduler examines this queue and allocates available cores to the programs. The scheduler aims at optimizing a scheduling objective, which we define later. 

Every program $P_i=\{b_{i1},\ldots,b_{in_i}\}$, with each block representing a sequence of instructions. In our analysis, a program is executed block by block as depicted in Fig.~\ref{fig:s1} and~\ref{fig:s2}.
%
Block's $b_j$ processing requirements, if executing over an $m$-type core are modeled by:
%
\begin{itemize}
\item $T_{jm}$ \-- block execution time;
\smallskip
\item $E_{jm}$ \-- block energy consumption,
\end{itemize}
%
These metrics are typically collected by profiling tools, such as $\textsc{Perf}$. Thus, we consider these metrics to be known prior to our analysis, e.g. like an example in Table~\ref{tab:demandExample}.


For a given program $P_i$, parameters $\mathsf{T}_m$ and $\mathsf{E}_m$ denote its total runtime and energy consumption over $m$-type core, which are computed over all its blocks by:
%
\begin{minipage}{0.46\columnwidth}
\smallskip
\begin{equation}
\mathsf{T}^i_m=\sum_{j=1}^{n}{T_{jm}}
\end{equation}
\smallskip
\end{minipage}%
\begin{minipage}{.5\columnwidth}
\smallskip
\begin{equation}
\mathsf{E}^i_m=\sum_{j=1}^{n}{E_{jm}}
\end{equation}
\smallskip
\end{minipage}


Considering execution requirements of every $P_1,\ldots,P_n$ programs are known, our key objective is to minimize the so-called ``energy-delay product'' metric \cite{Ratkovic2015}, which is denoted by $\uprho$ and computed by:
%
\begin{equation}
\uprho = \mathsf{T} * \mathsf{E}
\end{equation}
%
where $\mathsf{T}$ and $\mathsf{E}$ denote the aggregated runtime and energy consumption of all simultaneously executed programs:

\begin{minipage}{0.4\columnwidth}
\smallskip
\begin{equation}
\mathsf{T}=\sum_{i=1}^N{T^i_{\left(m_1\dots m_{n_i}\right)}}
\end{equation}
\smallskip
\end{minipage}%
\begin{minipage}{.4\columnwidth}
\smallskip
\begin{equation}
\mathsf{E}=\sum_{i=1}^N{E^i_{\left(m_1\dots m_{n_i}\right)}}
\end{equation}
\smallskip
\end{minipage}

where $\left(m_1\dots m_{n_i}\right)$ denotes a sequence of $n_i$ blocks of program $P_i$ executed over cores $m_j\in \{1,\ldots, \mathcal{M}\}$. In fact, we aim at determining all those sequences, which result in a program execution with an optimal trade-off between runtime and energy consumption.
%more details on computing $\uprho$ are provided in Section~\ref{sec:energyDelayProduct}.
%\todo{add motivational example probably referring to some figure}

We also study scheduling efficiency for minimum average response time and energy consumption in isolation in Chapter~\ref{chap:efficiencyMetrics}.

Additionally, for the analysis simplicity, we assume that:
%
\begin{itemize}
\item Granularity unit for our analysis is one program block;
\item Two programs cannot reuse the same core at a time;
\item Blocks within a given program execute sequentially;
\item Hardware overheads are insignificant (see Section~\ref{sec:performanceOverheads});
\item Block execution can be preempted and migrated between cores (see Section~\ref{sec:schedulingTypes}).
%All program blocks execute sequentially that is block $B_{i,{j+1}}$ starts when $B_{i,j}$ completes.
\end{itemize}
%
We aim at gradually relaxing these assumptions in the future.



\chapter{Methodology}
\label{chap:met}

Referencing other chapters \ref{chap:lr}, \ref{chap:met}, \ref{chap:impl}, \ref{chap:eval} and \ref{chap:conclusion}
\begin{longtable}{c|c}
\caption[This is the title I want to appear in the List of Tables]{Simulation Parameters} \label{table:thisimulation_params} \\
\hline
A & B  \\
\hline
\endfirsthead
\multicolumn{2}{@{}l}{} \\
\hline
A & B \\
\hline
\endhead
\hline
 \textbf{Parameter} & \textbf{Value}\\
 \hline
 Number of vehicles & $|\mathcal{V}|$\\
 \hline
 Number of RSUs & $|\mathcal{U}|$\\
 \hline
 RSU coverage radius & 150 m\\
 \hline
 V2V communication radius & 30 m\\
 \hline
 Smart vehicle antenna height & 1.5 m\\
 \hline
 RSU antenna height & 25 m\\
 \hline
 Smart vehicle maximum speed & $v_{max}$ m/s\\
 \hline
 Smart vehicle minimum speed & $v_{min}$ m/s\\
 \hline
 Common smart vehicle cache capacities & $[50, 100, 150, 200, 250]$ mb\\
 \hline
 Common RSU cache capacities & $[5000,1000,1500,2000,2500]$ mb\\
 \hline
 Common backhaul rates & $[75, 100, 150]$ mb/s\\
 \hline
\end{longtable}


\ldots
\chapter{Performance Metrics of Program Execution}
\label{chap:efficiencyMetrics}

A widely-established metric to balance time and energy performance of a program execution is Energy-Delay Product (EDP) \cite{Venkat2014, Ratkovic2015, Kumar2003, Gonzalez1996} defined by:
%
\begin{equation}
\uprho = \mathsf{E} * \mathsf{T}^\mathsf{RT}
\end{equation}
%
where $\mathsf{E}$ is energy consumed by a processing unit for a program execution and $\mathsf{T}^\mathsf{RT}$ is program response time discussed below. EDP metric is similar to power-delay product (PDP) used in digital electronics\footnote{https://wikipedia.org/wiki/Power-delay\_product}.

\section{Time performance metrics}
\label{sec:timePerformanceMetrics}

The major time performance metric is program response time denoted by $\mathsf{T}^\mathsf{RT}$, which is the time from its invocation until completion. For example, in a non-preemptive scenario of Fig.~\ref{fig:nonPreemptiveExecution}, the program response time is 12, and in a preemptive in Fig.~\ref{fig:preemptiveExecution} it is 15 time units. In turn, $\mathsf{T}^\mathsf{RT}$ is subject to various other metrics, including preemption and migration overheads discussed in Section~\ref{sec:performanceOverheads}, latency and execution time, discussed next.

Consider a 3-blocks program execution in Fig.~\ref{fig:preemptiveExecution}. 
The time since program's release until its actual start is latency ($\Delta t^{\mathsf{l}}$), which is 2 time units in the figure. Next 4 time units are program's first block execution time, which is the time since a block start until its completion. The sum of all blocks execution times is program execution time ($t^{\mathsf{exec}}$). While, the sum of blocks preemption times is program preemption time ($t^{\mathsf{pm}}$), which is the time when the program is preempted by other programs. 

With these metrics we express program response time as:
%
\begin{equation}
\mathsf{T}^\mathsf{RT}=t^\mathsf{c}-t^\mathsf{r}=\Delta t^\mathsf{l}+\Delta t^\mathsf{pm}+\Delta t^\mathsf{exec}*1.25
\end{equation}
%
where $t^\mathsf{r}$ and $t^\mathsf{c}$ are program release and completion times.
We note that performance overheads are not yet considered. Instead, we assume that they take at most 25\% of program execution time.

\begin{figure}
\centering
\begin{subfigure}{.7\columnwidth}
\includegraphics[width=\columnwidth]{figs/nonPreemptiveExecution.pdf}
\caption{Non-preemptive execution}
\label{fig:nonPreemptiveExecution}
\end{subfigure}
\begin{subfigure}{.7\columnwidth}
\includegraphics[width=\columnwidth]{figs/preemptiveExecution.pdf}
\caption{Preemptive execution at blocks completion}
\label{fig:preemptiveExecution}
\end{subfigure}
\caption{Time metrics of an N-blocks program execution}
\label{fig:executionMetrics}
\end{figure}

For our toy example, below in Fig.~\ref{fig:twoSchedulesExample} we compute average program response and execution times for two schedules. Notice that schedule S1 has higher average response time, but lower average execution time than S2. Also, S1 preemption time is twice larger than in S2. %In our extended example in Fig.~\ref{fig:s1PreemptionMigration} the total preemption times of Prog. P1, P2 and P3 are 3, 6 and 0 time units correspondingly.

\section{Energy performance metrics}
\label{sec:energyConsumption}

Nowadays energy efficiency deserves attention and is considered equally with program response time. However, unlike response time, energy consumption is difficult to measure. Typical direct measurement approaches with specialized devices, such as wattmeter and multimeter, which are attached to an operating hardware, are often impracticals. This motivates the development of theoretical techniques to model energy consumption. This modeling mostly relies on the analysis of different program execution events collected from the following sources\footnote{https://www.brendangregg.com/perf.html}:
\begin{itemize}
\item Hardware: Performance monitor counters (PMCs);
\item Operating system: Kernel counters and tracepoints;
\item Tracing: Dynamic instrumentation of user-level software.
\end{itemize}
Specifically, PMC represented as registers collect statistics of CPU and memory units. While tracing in software collects the statistics on a higher level program execution. Then, this statistics, such as number of clock cycles, integer or floating point instructions, branch mispredictions, cache misses, page faults, context switches and migrations, is aggregated into a value representing energy consumption of a program execution. Some studies expect that a modeling error is within 5-10\% compared to the direct measurements\cite{Joseph2001, Brooks2000} or cycle-accurate simulation \cite{Li2003}.

Energy consumption denoted by $\mathsf{E}$ is defined by:
%
\begin{equation}
\mathsf{E} = \mathsf{P} * \mathsf{T}^\mathsf{RT}
\end{equation}
%
where $\mathsf{P}$ is average power in Watts used by hardware for a program execution, and $\mathsf{T}^\mathsf{RT}$ is program response time in seconds. Hence, $\mathsf{E}$ is measured in Watt-seconds, which are actually Joules denoted by $\mathsf{J}$.

Moreover, various other energy performance metrics are available, such as Energy per Instruction (EPI), Energy per Operation (EPO), which are however not yet considered in our analysis. 
	
	
\section{Optimization objective}
\label{sec:energyDelayProduct}
\label{sec:optimizationObjective}

Our key optimization objective is a so-called Energy-Delay Product (EDP) metric \cite{Ratkovic2015, Gonzalez1996}, which we denote by $\uprho$. It aggregates both time and energy performance metrics of all executing programs

For an individual program EDP is computed by:
%
\begin{equation}
\uprho = \mathsf{E} * \Delta t^\mathsf{RT}
\end{equation}
%
Here, both energy and time are considered equally, while in alternatives to $\mathsf{EDP}$, such as $\mathsf{ED^2P}$ or $\mathsf{ED^3P}$, time is critical:
%
\begin{equation}
\uprho^2 = \mathsf{E} * \Delta t^{\mathsf{RT}^2}
\end{equation}
%
%
\begin{equation}
\uprho^3 = \mathsf{E} * \Delta t^{\mathsf{RT}^3}
\end{equation}
%

Consider in Fig.~\ref{fig:EDPExample}, which depicts energy consumption vs. average program response of three abstract schedules. Schedule 1, represented as a dark square, has EDP of 300 Joules-second, which is indicated by the hyperbolic dashed-line. While schedules 2 and 3, represented by two lighter circles, have equal EDP of 150. Note the difference in ratios of times to energy between schedule 1-2 and 3-2. Compared to schedule 1, the execution time of schedule 2 increased less than energy consumption decreased. While between schedules 2 and 3 these time/energy ratios are the same. Hence, schedules 2 and 3 are equal and better than schedule 1 from the EDP-perspective.

The EDP metric is extensively used later in our analysis starting from Section~\ref{sec:schedulingProcedure}.

%Also, we compute $\mathsf{EDP}$ in Fig.~\ref{fig:programsEDP} for each program of both schedules S1 and S2 in Fig.~\ref{fig:twoSchedulesExample}. Note the difference in 92 and 71\% for Prog. P2 and P3, which indicates that schedule S2 is significantly more efficient than S1.


\begin{figure}
\centering
\includegraphics[width=.7\columnwidth]{figs/EDPExample.pdf}
\caption{Trade-off between response time and energy consumption. An optimal line (Sched. 2 and 3) has better trade-off.}
\label{fig:EDPExample}
\end{figure}

%\section{Existing heterogeneous schedulers}
%\label{sec:existingSchedulers}

%\todo{To list 2-3 publicly available implementations of a heterogeneous scheduler, discussing their flaws, consideration of overheads, choice of scheduling objective and energy-awareness}



\chapter{Optimized Heterogeneous Scheduling\\Driven by State-Transition Graphs}
\label{chap:optimizedHeterogeneousScheduling}

We are now ready to describe our optimized heterogeneous scheduler. First we provide the necessary background on state-transition graphs, which are the core of our scheduler, with Section~\ref{sec:programStg} discussing the case of a single program, Section~\ref{sec:joint_graph} describing the composition of graphs for multiple programs, and Section~\ref{sec:schedulingProcedure} summarizes the overall scheduling decisions mechanism. 


\section{Step 1: Individual program schedules}
\label{sec:programStg}

To represent all possible execution schedules for a given program we use a state-transition graph (STG). Graph states correspond to different allocation scenarios of heterogeneous hardware to program blocks. A simple graph example is provided in Fig.~\ref{fig:P2STG}, which corresponds to the $P_2$ program with execution requirements from Table~\ref{tab:demandExample}. Next we formalize the notion of such a state-transition graph.

\begin{figure*}
\center
\includegraphics[width=0.95\textwidth]{figs/P2STG.pdf}
\caption{The state-transition graph of the program $P_2$ from  Table~\ref{tab:demandExample}}
\label{fig:P2STG}
\end{figure*}

Consider a program of $n$ blocks, denoted by $P = [b_1,\ldots,b_n]$.
Formally, its state-transition graph is defined according to the next principles.

\textbf{Graph state} is a tuple:
%
\begin{equation}
x = \left(m, \, j^\mathsf{block}, \, t^\mathsf{start}, \, t^\mathsf{end} \right),
\label{eq:graphState}
\end{equation}
%
where:
%
\begin{enumerate}
\item $m \in \{1,\ldots,\mathcal{M}\}$ is an allocated core type;
\item $j^\mathsf{block} \in \{1,\ldots,n\}$ is index of an executing block;
\item $t^\mathsf{start}, t^\mathsf{end}$ are start and end times of an interval allocated for block execution, counted from the initial state start.
\end{enumerate}


\textbf{Graph transition} defines, for a given state $x$, a set $\hat{X}(x)=\{\hat{x}_1, \ldots, \hat{x}_\mathcal{M}\}$ of its reachable successors\footnote{For briefness, the term ``successors" refers to ``neighbor successors"}, which is computed by:
%
\begin{equation}
\begin{aligned}
\hat{x}_m \in \hat{X}(x) \\
\forall m \in \{1,\ldots,\mathcal{M}\}
\end{aligned}
\iff
%
\begin{cases}
\hat{j}^\mathsf{block} = j^\mathsf{block} +1\\
\hat{t}^\mathsf{start} = t^\mathsf{end}\\
\hat{t}^\mathsf{end} = \hat{t}^\mathsf{start} + \left(t^\mathsf{end} - t^\mathsf{start}\right)
\end{cases}
\end{equation}
%
Set $\hat{X}$ construction is similar to \cite{Burmyakov2015}.

The execution requirements for time $t_{jm}$ and energy use $e_{jm}$ are derived from $t^\mathsf{start}$ and $t^\mathsf{end}$ as following:
%

\begin{minipage}{0.35\columnwidth}
\smallskip
\begin{equation}
t_{jm} = t^\mathsf{end} - t^\mathsf{start},
\end{equation}
\smallskip
\end{minipage}%
\begin{minipage}{.35\columnwidth}
\smallskip
\begin{equation}
e _{jm}= \frac{t^\mathsf{end} - t^\mathsf{start}}{T_{jm}}*E_{jm}
\end{equation}
\smallskip
\end{minipage}
%
Above, $T_{jm}$ and $E_{jm}$ are initial time and energy requirements of block $j^\mathsf{block}$ to be executed over an $m$-type core.  


\textbf{Graphical notation} for visualisation, also shown in Fig. \ref{fig:P2STG}, is the following:
%
\begin{itemize}
\item States are represented by rectangles with curved angles;
\item The state at the top having an incoming arrow is a program initial state, which precedes execution start;
\item Curved lines between states are transitions corresponding to some scheduling events;
\item $\delta^{\mathsf{overhead\_type}}$, which is placed near a transition arrow, denotes some performance overhead caused by this transition, e.g. due to migrations between cores.
\end{itemize}
%
For more descriptive representation, we also show $t_{jm}$ and $e_{jm}$, which are derived from math notation above.

%In turn, transitions between states represent various scheduling events, including
%%
%\begin{itemize}
%\item A core waking up or turning down;
%\item Execution start or completion of a block;
%\item Migration between processing units;
%\item Preemption by other program, in case of preemptive scheduling.
%\end{itemize}

The state-transition graph of an individual program $P$ assumes an entire hardware platform to be dedicated to the execution of exactly this program $P$ with no concurrent competitors considered. In fact, the state-transition graph is a very flexible and extensible method for modeling various scheduling problems~\cite{Burmyakov2021}.

\textbf{A program schedule} is  a graph branch from the initial state to some leaf state. In case of $k$ branches a set of schedules $\mathcal{S}$ for an $n$-block program's STG is defined as following:
%
\begin{equation}
\mathcal{S}=\{ s_1,s_2,\ldots,s_k\}
\end{equation}
%
Considering that every graph state corresponds to a certain program block execution, we say that some schedule $s_k$ is a sequence of block states:
\begin{equation}
s_k = \left[x_k(b_1),x_k(b_2),\ldots,x_k(b_n)\right]
\end{equation}
%
where each block state $x_k(b_j)$ is defined by~(\ref{eq:graphState}).

  


\section{Step 2: A merged state-transition graph}
\label{sec:joint_graph}

Typically multiple programs execute concurrently over a shared heterogeneous platform. In this case the scheduling problem is to determine an efficient allocation of heterogeneous processing units to those programs, according to chosen scheduling objectives. Moreover, it must be considered that programs arrive asynchronously and their execution over the same processing unit yields different performance gain: for example, a matrices multiplication program executes drastically faster over a GPU, while strongly sequential non-parallelisable programs typically do not benefit from GPU use in any way, despite its computational power and provided parallelism.

For our efficient scheduling we model possible concurrent executions of given programs. For such modeling we merge STGs of individual programs into a so-called joint state-transition graph (jSTG). An example of a simple jSTG is depicted in Fig.~\ref{fig:jSTGExample}, and its formal description is given below.

\textbf{Graphs state} of a jSTG, called a system state, is a set of program states $\mathcal{P} = \{y_1,\ldots,y_N\}$. Each program state $y_k$ is a tuple similar to a state $x$ of an individual program's STG described above, i.e. describes an execution of a program block $x.j^{block}$. The state represents an entire or partial block $y_k.j^{block}$ execution. In case of the entire execution, a program state is the same as a state in its STG.  In case of partial execution, a program state spreads across $l$ consequent system states $Y$ such that the core type $y.m$ is the same $\forall y \in Y$ and $\sum_{k=1}^l{y_k.t^{end}-y_k.t^{start}}=T_{y_k.j^{block},m}$, i.e. the overall time allocated to block $j$ execution equals to required time for an $m$ type core.

Overall, the jSTG state $y_k(\mathcal{P})$ of a programs state set $\mathcal{P}=\{x_1,\ldots,x_N\}$ is defined as following:
%
\begin{equation}
\begin{aligned}
y_k(\mathcal{P}) = \{x_1,\ldots,x_N\} \iff x_a.m_{l1} \neq x_b.m_{l2} \\
\forall l1,l2 \in \{1,\ldots,\mathcal{M}\}\ |\ l1 \neq l2
\end{aligned}
\label{eq:jGraphState}
\end{equation}
%

%\textbf{Graph transition} defines, for a given state $\mathcal{p}$ of $N$ programs, a set $\hat{P}(p)$

\textbf{Graphical notation} of jSTG depicted in Fig.~\ref{fig:jSTGExample} provides:
\begin{enumerate}
\item Top black circle \-- the initial state;
\item Dashed rectangles \-- combined program states\footnote{The programs set excludes programs without an allocated core};
\item Horizontal dashed lines \-- programs arrivals;
\item ``Idle" keywords \-- no-program state;
\item Paths from initial state to graph leafs \-- schedules..
\end{enumerate}
The color of program states distinguishes programs, which corresponds to programs colors in Fig.~\ref{fig:twoSchedulesExample}.  The example illustrates schedules S1 and S2 from Fig.~\ref{fig:twoSchedulesExample}.

\begin{figure*}
\center
\includegraphics[width=.9\textwidth]{figs/jSTG.pdf}
\caption{A joint state-transition graph with schedules from Fig.~\ref{fig:twoSchedulesExample}. The case of non-preemptive scheduling.}
\label{fig:jSTGExample}
\end{figure*}

The joint state-transition graph of concurrent programs assumes:
%
\begin{itemize}
\item An entire heterogeneous platform is available;
\item A single core executes at most one program at a time;
\item Performance overheads are insignificant.
\end{itemize}
%
The example with three programs is directly extendable to multiple concurrent programs by analogy.

The graph in Fig.~\ref{fig:jSTGExample} models the execution of three programs $P_1$, $P_2$, $P_3$, with their parameters listed in Table~\ref{tab:demandExample} over $m=2$ core types. Observe that programs arrive asynchronously at time instants $t=0,\,1,$ and $3$.

\textbf{A schedule} for a joint-STG represents a sequence of graph system states. Where a system state is a set of program states defined above. A set of such schedules $\mathcal{S}^\mathsf{joint}$ is defined by:
%
\begin{equation}
\mathcal{S}^\mathsf{joint}=\{ s_1^\mathsf{joint}\ldots,s_k^\mathsf{joint}\}
\end{equation}
%
Considering a single schedule $s^\mathsf{joint}_k$ to be a sequence of $l$ system states for a set of programs $\mathcal{P}=\{p_1,\ldots, p_N\}$. Such a sequence is defined as follows:
%
\begin{equation}
s^\mathsf{joint}_k = \left[y_{1,k}(\mathcal{P}),\ldots,y_{l,k}(\mathcal{P})\right]
\end{equation}
%
where $y_l,k(\mathcal{P})$ is a graph system state defined in~(\ref{eq:jGraphState}).




%\section{Step 3: Allocation of processing units}
%\label{sec:optimized_scheduling_decisions}


\section{Scheduling procedure: Putting the pieces together}
\label{sec:schedulingProcedure}

\begin{figure}
\center
\includegraphics[width=.8\columnwidth]{figs/compilerAssistedSchedulingConcept.pdf}
\caption{Proposed extension to heterogeneous scheduling}
\label{fig:compilerAssistedSchedulingConcept}
\end{figure}

We are finally ready to formalize the procedure of our optimized heterogeneous scheduling. Its key steps are outlined in Fig.\ref{fig:compilerAssistedSchedulingConcept} and are the following:

\begin{itemize}
\itemindent=26pt
\item[\emph{Step 1:}] A compiler generates STGs from programs source code (Section~\ref{sec:programStg});
\item[\emph{Step 2:}] A scheduler merges STGs of programs to be concurrently executed (Section~\ref{sec:joint_graph});
\item[\emph{Step 3:}] The scheduler allocates processing units to programs based on the constructed joint-STG as discussed below);
\end{itemize}
%
To make efficient heterogeneous sheduling decisions, a scheduler analyzes total response times and energy consumptions of different execution schedules\footnote{The schedules examination procedure can be optimized with state-space pruning techniques \cite{Burmyakov2022}, which avoids non-optimal transitions in a state-transition graph},
%
which are illustrated below in Fig.~\ref{fig:P2STG}. An operating system scheduler then examines summaries of these schedules aiming at an optimal trade-off between schedule response time and energy consumption, which is discussed in Section~\ref{sec:energyDelayProduct}. 


\nocite{*}
\printbibliography[heading=bibintoc,title={Список использованной литературы}]
\end{document}

